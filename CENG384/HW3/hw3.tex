\documentclass[10pt,a4paper, margin=1in]{article}
\usepackage{fullpage}
\usepackage{amsfonts, amsmath, pifont}
\usepackage{amsthm}
\usepackage{graphicx}


\usepackage{geometry}
 \geometry{
 a4paper,
 total={210mm,297mm},
 left=10mm,
 right=10mm,
 top=10mm,
 bottom=10mm,
 }
 
\usepackage{tkz-euclide}
\usepackage{tikz}
\usepackage{pgfplots}
\pgfplotsset{compat=1.13}

\begin{filecontents}{q9.dat}
 n   xn 
 -8   1
 -7  -0.5
 -6  0
 -5  -0.5
 -4	 1
 -3  -0.5
 -2  0
 -1  -0.5
 0   1
 1   -0.5  
 2   0
 3   -0.5
 4   1 
 5   -0.5
 6   0
 7   -0.5
 8   1
\end{filecontents}



%magnitude
\begin{filecontents}{q12.dat}
 n   xn 
 -8  0.75
 -7  1.11 %kok(5)/4  
 -6  0.25
 -5  1.11 %kok(5)/4  
 -4	 0.75
 -3  1.11 %kok(5)/4  
 -2  0.25
 -1  1.11 %kok(5)/4  
 0   0.75
 1   1.11 %kok(5)/4  
 2   0.5
 3   1.11 %kok(5)/4  
 4   0.75
 5   1.11 %kok(5)/4  
 6   0.5
 7   1.11 %kok(5)/4  
 8   0.75
\end{filecontents}


%phase
\begin{filecontents}{q11.dat}
 n   xn 
 -8  -36.8
 -7  -26.5
 -6  -14.03
 -5  26.5
 -4	 -36.8
 -3  -26.5
 -2  -14.03
 -1  26.5
 0   36.8
 1   -26.5  
 2   26.5
 3   -26.5
 4   36.8 
 5   -26.5
 6   26.5
 7   -26.5
 8   36.8
\end{filecontents}

 % Write both of your names here. Fill exxxxxxx with your ceng mail address.
 \author{
  Sen, Ali\\
  \texttt{e2264661@ceng.metu.edu.tr}
  \and
  Sahin, Ismail\\
  \texttt{e2264653@ceng.metu.edu.tr}
}
\title{CENG 384 - Signals and Systems for Computer Engineers \\
Spring 2018-2019 \\
Written Assignment 3}
\begin{document}
\maketitle



\noindent\rule{19cm}{1.2pt}

\begin{enumerate}

\item 
    \begin{enumerate}
    % Write your solutions in the following items.
    \item %write the solution of q1a
    Period N = 4,\\ $W_0 = 2\pi/4 = \pi / 2$ \\
    $ a_k = 1/N \sum_{n=0}^{3} X[n]e^{-jW_0kn}$ \\
    $ a_k=0 + 1/4e^{-jW_0k}+ 1/2e^{-jW_0k2} + 1/4e^{-jW_0k3}$ \\
    $ a_k=1/2e^{-jW_0k2} + 1/4(e^{-jW_0k} + e^{-jW_0k3})$ \\
    $ a_k=1/2(cos(W_0k2)-jsin(W_0k2) ) + 1/4(cos(W_0k)-jsin(W_0k) + cos(W_0k3)-jsin(W_0k3))$ \\
    Plug in $W_0 = \pi / 2$\\
    $ a_k=1/2(cos(\pi k)-jsin(\pi k) ) + 1/4(cos((\pi/2)k)-jsin((\pi/2)k) + cos((3\pi/2)k)-jsin((3\pi/2)k))$ \\ \\
    $k=0   \qquad  a_0 = 1$ \\
    $k = 1 \qquad  a_1= -1/2$ \\
    $k = 2 \qquad  a_2= 0$ \\
    $k = 3 \qquad  a_3= -1/2$ \\
    and because it is a symmetric signal \\ 
    $a_1 = a_{-1} \quad and \quad a_2 = a_{-2} \quad and \quad so \quad on..$
    


\begin{figure} [h!]
    \centering
    \begin{tikzpicture}[scale=1.0] 
      \begin{axis}[
          axis lines=middle,
          xlabel={$k$},
          ylabel={$\boldsymbol{a_k}$},
          xtick={-8,-7,-6,..., -1, 0,1,2, ..., 8},
          ytick={-2, -1, ..., 2},
          ymin=-2, ymax=2,
          xmin=-8, xmax=8,
          every axis x label/.style={at={(ticklabel* cs:1.05)}, anchor=west,},
          every axis y label/.style={at={(ticklabel* cs:1.05)}, anchor=south,},
          grid,
        ]
        \addplot [ycomb, black, thick, mark=*] table [x={n}, y={xn}] {q9.dat};
      \end{axis}
    \end{tikzpicture}
    \caption{ constants}
    \label{fig:q3}
\end{figure}
    
    
    
    
    
    
    
    
    \item %write the solution of q1b
	Period N = 4,\\ $W_0 = 2\pi/4 = \pi / 2$ \\
    $ a_k = 1/N \sum_{n=0}^{3} X[n]e^{-jW_0kn}$ \\
    $ a_k = (1/4)e^{-jkW_0} + (1/4)e^{-jkW_02}$ \\
    $ a_k = (1/4)e^{-jkW_0} + (1/4)e^{-jkW_02}$ \\
    $ a_k=1/2(cos(W_0k2)-jsin(W_0k2) ) + 1/4(cos(W_0k)-jsin(W_0k)) $ \\
    Plug in $W_0$ \\
	$ a_k=1/2(cos(k \pi)-jsin(k \pi) ) + 1/4(cos((\pi /2)k)-jsin((\pi /2)k)) $ \\ \\
	
	$k=0   \qquad  a_0 = 3/4$ \\
    $k = 1 \qquad  a_1= (1/2)((-j/2)-1)$ \\
    $k = 2 \qquad  a_2= 1/2$ \\
    $k = 3 \qquad  a_3= (1/2)((j/4)-1)$ \\ \\

    $k = -1 \qquad  a_{-1}= (1/2)((j/4)+1)$ \\
    $k = -2 \qquad  a_{-2}= -1/4$ \\
    $k = -3 \qquad  a_{-3}= (1/2)((-j/4)+1)$ \\

    

%magnitude
	
\begin{figure} [h!]
    \centering
    \begin{tikzpicture}[scale=1.0] 
      \begin{axis}[
          axis lines=middle,
          xlabel={$k$},
          ylabel={$\boldsymbol{magnitude}$},
          xtick={-8,-7,-6,..., -1, 0,1,2, ..., 8},
          ytick={-2, -1, ..., 2},
          ymin=-2, ymax=2,
          xmin=-8, xmax=8,
          every axis x label/.style={at={(ticklabel* cs:1.05)}, anchor=west,},
          every axis y label/.style={at={(ticklabel* cs:1.05)}, anchor=south,},
          grid,
        ]
        \addplot [ycomb, black, thick, mark=*] table [x={n}, y={xn}] {q12.dat};
      \end{axis}
    \end{tikzpicture}
    \caption{ magnitude}
    \label{fig:q3}
\end{figure}
    
    
    
    
    
%phase
    
    	
\begin{figure} [h!]
    \centering
    \begin{tikzpicture}[scale=1.0] 
      \begin{axis}[
          axis lines=middle,
          xlabel={$k$},
          ylabel={$\boldsymbol{phase}$},
          xtick={-8,-7,-6,..., -1, 0,1,2, ..., 8},
          ytick={-40,-30,-20,-10,0,10 ,20,30,40},
          ymin=-40, ymax=40,
          xmin=-8, xmax=8,
          every axis x label/.style={at={(ticklabel* cs:1.05)}, anchor=west,},
          every axis y label/.style={at={(ticklabel* cs:1.05)}, anchor=south,},
          grid,
        ]
        \addplot [ycomb, black, thick, mark=*] table [x={n}, y={xn}] {q11.dat};
      \end{axis}
    \end{tikzpicture}
    \caption{ phase}
    \label{fig:q3}
\end{figure}
    
    
    
we find phase by using arctan values of complex constanst \\
    
    
	
    \end{enumerate}





\item %write the solution of q2



a)$N=4, W_0=\frac{2{\pi}}{4}=\frac{{\pi}}{2}$ \\
b)From  $-3$ to $0$ there are $2$ periods. Let's take one period\\
$x[0]+x[1]+x[2]+x[3]=4$\\
c) add $1$ period to $a_{-3}$, we get $a_1$\\
Subtract $3$ period from $a_{15}^*$, we get $a_3^*$\\
$a_1=a_3^*$ \ \ $a_1=x+yj$ \ \ $a_3=x-yj$ \\
$|a_1-a_3|=1$ 
d)One of the coefficients is $0$\\
e)$e^{-j{\pi}k/2}=cos({\pi}k/2)-jsin({\pi}k/2)$\\
$e^{-j{\pi}3k/2}=cos(3{\pi}k/2)-jsin(3{\pi}k/2)$\\
$=cos((3{\pi}k/2)-2{\pi}k)-jsin((3{\pi}k/2)-2{\pi}k)$\\
$=cos(-{\pi}k/2)-jsin(-{\pi}k/2)$\\
$=cos({\pi}k/2)+jsin({\pi}k/2)$\\ \\
$\sum_{k=0}^{3}(e^{-j{\pi}k/2}+e^{-j{\pi}3k/2})=\sum_{k=0}^{3}x[k](2cos({\pi}k/2))=4$\\ \\
$2x[0]-2x[2]=4$\\
$x[0]-x[2]=2$\\ \\
$a_k=\frac{1}{N}\sum_{n=<N>}^{}x[n]e^{-jkw_0n}$\\ \\
$a_k=\frac{1}{4}x[0]+\frac{1}{4}x[1]e^{-jkw_0}+\frac{1}{4}x[2]e^{-jkw_02}+\frac{1}{4}x[3]e^{-jkw_03}$\\ \\
$a_0=\frac{1}{4}x[0]+\frac{1}{4}x[1]+\frac{1}{4}x[2]+\frac{1}{4}x[3]$\\ \\
$a_1=\frac{1}{4}x[0]+\frac{1}{4}x[1]e^{-j{\pi}/2}+\frac{1}{4}x[2]e^{-j{\pi}}+\frac{1}{4}x[3]e^{-j3{\pi}/2}$\\ \\
$a_2=\frac{1}{4}x[0]+\frac{1}{4}x[1]e^{-j{\pi}}+\frac{1}{4}x[2]e^{-j2{\pi}}+\frac{1}{4}x[3]e^{-j{\pi}}$\\ \\
$a_3=\frac{1}{4}x[0]+\frac{1}{4}x[1]e^{-j3{\pi}/2}+\frac{1}{4}x[2]e^{-j{\pi}}+\frac{1}{4}x[3]e^{-j{\pi}/2}$\\ \\ \\
$a_0=\frac{1}{4}(x[0]+x[1]+x[2]+x[3])=1$ from b we know\\
the inside of the paranthesis are 4. $a_0=1$\\ \\
$a_2=\frac{1}{4}x[0]+\frac{1}{4}x[1](cos(\pi)-jsin(\pi))+\frac{1}{4}x[2](cos(2\pi)-jsin(2\pi))$\\ \\$+\frac{1}{4}x[3](cos(\pi)-jsin(\pi))$\\ \\
$a_1$ or $a_3$ won't be $0$ because they are conjugate \\
and complex number j is $1$ as magnitude.\\ \\
$a_2=\frac{1}{4}x[0]-\frac{1}{4}x[1]+\frac{1}{4}x[2]-\frac{1}{4}x[3]=0$\\ \\
$\frac{1}{4}(x[0]-x[1]+x[2]-x[3])=0$\\ \\
From e $2x[0]+2x[2]=4$\\
$x[0]+x[2]=2$\\
$x[0]-x[2]=2$\\
$x[0]=2$, $x[2]=0$\\ \\
$a_1=\frac{1}{4}x[0]+\frac{1}{4}x[1](cos(\pi/2)-jsin(\pi/2))$\\ \\
$+\frac{1}{4}x[2](cos(\pi)-jsin(\pi))+\frac{1}{4}x[3](cos(3\pi/2)-jsin(3\pi/2))$\\ \\
$a_1=\frac{1}{4}x[0]-\frac{j}{4}x[1]-\frac{1}{4}x[2]+\frac{j}{4}x[3]$\\ \\ \\
$a_3=\frac{1}{4}x[0]+\frac{1}{4}x[1](cos(3\pi/2)-jsin(3\pi/2))$\\ \\
$+\frac{1}{4}x[2](cos(\pi)-jsin(\pi))+\frac{1}{4}x[3](cos(\pi/2)-jsin(\pi/2))$\\ \\
$a_3=\frac{1}{4}x[0]+\frac{j}{4}x[1]-\frac{1}{4}x[2]-\frac{j}{4}x[3]$\\ \\
from c $|a_1-a_3|=|\frac{-j}{2}x[1]+\frac{j}{2}x[3]|=1$\\ \\
$|x[3]-x[1]|=2$\\
There are two possibilities $x[3]-x[1]=2$ and $x[1]-x[3]=2$ \\
Let's take one of them \\
From b:\\
$2+0+x[1]+x[3]=4$\\
$x[1]+x[3]=2$\\
$x[1]-x[3]=2$\\
Result:\\
$x[0]=2$\\
$x[1]=2$\\
$x[2]=x[3]=0$\\ \\ 













\item %write the solution of q3     
$x(t) = h(t) *(x(t)+r(t))$ \\
$X(jw) = H(jw)((X(jw)+R(jw)))$ \\
$X(jw) = H(jw)X(jw) + H(jw)R(jw)$ \\
$R(jw) = 0$ \\
$H(jw) = 1$ \\
$ h(t) = (1/2\pi) \int_{-\infty}^{\infty} H(jw)e^{jwt}dw = (1/2\pi)\int_{-2k\pi/T}^{2k\pi/T} e^{jwt}dw = (1/2\pi)(e^{jwt }/jt)$ \\
$(1/2\pi) ( (e^{2\pi jkt/T } / jt) - (e^{-2\pi jkt/T } / jt))$ \\
 $(1/ \pi t) - sin(2\pi kt)/T$ \\ \\



\item 
    \begin{enumerate}
    \item %write the solution of q4a
    $y''(t) + 5y'(t) + 6y(t) = 4x'(t) + x(t)$ \\
    $x(t) = e^{jwt} $ the fourier transform of x(t) is $y(t) = H(jw)e^{jwt}$\\
    $y'(t) = (jw)H(jw)e^{jwt}$ \\
    $y''(t) = (jw)^2H(jw)e^{jwt}$ \\
    $x'(t) = (jw)e^{jwt}$ \\ \\
    $(jw)^2H(jw)e^{jwt}+5(jw)H(jw)e^{jwt}+6H(jw)e^{jwt} = 4(jw)e^{jwt}+e^{jwt}$ \\
    $H(jw) =( (4(jw)+1) / ((jw)^2 + 5(jw) +6) ) = A / (jw+3) + B / (jw+2)$ \\
    A+B = 4 from the (jw)'s constant\\
    2A + B = 1 \\
    A = 11 \\
    B = -7 \\
    $H(jw) = (11/(3+jw)) -(7 / (2+jw))$\\
    
    
    \item %write the solution of q4b
    By using fourier transform table we can find the impulse response of the system with the help of frequency response
    $ H(t) = 11e^{-3t} u(t) -7e^{-2t}u(t) = (11e^{-3t} -7e^{-2t})u(t)$
    \item %write the solution of q4c
    Y(jw) = H(jw)X(jw) \\
    $x(t) = (1/4) e^{-t/4}u(t) $ by using fourier transform table  $ X(jw)= (1/4) / ((1/4)+(jw))$ \\
    $H(jw) = ( (4((jw)+1/4)) / ((jw)^2 + 5(jw) +6) )$\\
    $H(jw)X(jw) = 1/((jw)^2 + 5(jw) +6)$ \\
    $1 = (A / (jw+3)) + (B / (jw+2))$ \\
    A+ B = 0\\
    2A + 3B = 1\\
    A = -1 \\
    B = 1 \\
    $Y(jw) = (-1/(jw+3)) + (1 / (jw+2)) $ \\
    From table again \\
    $y(t) = (-e^{-3t} + e^{-2t})u(t)$
    
    \end{enumerate}






\end{enumerate}
\end{document}

